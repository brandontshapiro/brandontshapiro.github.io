\documentclass{moderncv}

\moderncvstyle{classic}
\moderncvcolor{black} %black?
\usepackage[margin=2.5cm]{geometry}
%\usepackage{full page} %not as good because it doesn't adjust the bottom of the page
\usepackage{palatino}
%\usepackage{l3backend}

\usepackage{amssymb}
\usepackage{amsmath}


%Get rid of the pesky icons
\renewcommand*{\addresssymbol}       {}
\renewcommand*{\mobilephonesymbol}   {}
\renewcommand*{\emailsymbol}         {}
\renewcommand*{\homepagesymbol}      {}


\usepackage{color}
\newcommand{\hlink}[2]{\textcolor{blue}{\href{#2}{#1}}}

%contact info stuff
\name{Brandon T.}{Shapiro}
%\phone[mobile]{(301) 806-4186}              
\extrainfo{
\hlink{brandonshapiro@virginia.edu}{mailto:brandonshapiro@virginia.edu}\\
\hlink{brandontshapiro.github.io}{https://brandontshapiro.github.io}
}

\begin{document}

%TITLE 
\vspace*{-1cm}

\makecvtitle

\vspace*{-1cm}



\section{Research Interests}

Combinatorial approaches to topology, geometry, $K$-theory, higher categories, and applied category theory.

%Higher categories, algebraic K-theory, applied category theory, combinatorial homotopy theory. 
%
%\vspace{.2cm}
%
%I am especially interested in algebraic higher category structures with arbitrary cell shapes, the $K$-theory of finite sets, combinatorial and algebraic models for the homotopy theory of spaces and $(\infty,n)$-categories, generalized frameworks for $K$-theory, and polynomial functors as a formalism for open dynamics and category theory.



\section{Employment}

\cventry{2023-Present}{Whyburn Research Associate and Lecturer}{University of Virginia}{Charlottesville, VA.}{}{}

\cventry{2022-2023}{Research Associate}{Topos Institute}{Berkeley, CA,}{AFOSR grant.}{}%{Research the theory of polynomial functors with David Spivak.}

\cventry{2015}{Formal Methods Intern}{Draper Laboratories}{Cambridge, MA,}{DARPA grant.}{}%{Research the theory of polynomial functors with David Spivak.}


%Education
\section{Education}

\cventry{2022}{PhD in Mathematics}{Cornell Unversity}{Advisor: Inna Zakharevich.}{}{Thesis: Shape Independent Category Theory.}

%\vspace{.1cm}

%\cventry{2017-Present}{Teaching Certificate Program}{Cornell Unversity}{Mathematics Department}{}{}

%\vspace{.1cm}

\cventry{2019}{Master of Science in Computer Science}{Cornell Unversity.}{}{}{}%{Supervised by Dexter Kozen.}

%\vspace{.1cm}

\cventry{2017}{Bachelor of Arts with Highest Honors in Mathematics}{Brandeis University.}{}{}{}%{Key Courses: Homotopy Theory, Algebra, Differential Geometry. 4.0/4.0 GPA.}

%\vspace{.1cm}

\cventry{2017}{Bachelor of Science in Computer Science}{Brandeis University.}{}{}{}%{Key Courses: Modal Logic in Language, Computing Theory, Database Systems. 4.0 GPA.}

%\vspace{.1cm}

\cventry{2017}{Bachelor of Arts in Physics}{Brandeis University.}{}{}{}%{Key Courses: General Relativity, Classical Mechanics, Statistical Mechanics. 3.97 GPA.}

%\vspace{.1cm}

\cventry{2016}{Brandeis India Science Scholars Program}{Indian Institute of Science.}{}{}{}%{Key Courses: Computational Topology, Mathematical Quantum Mechanics, Analysis}





%Awards
\section{Honors and Awards}

\cvitem{2021}{\textbf{B\"attig Prize for Excellence and Promise in Mathematics}, Cornell University.}

\cvitem{2017-2021}{\textbf{National Defense Science \& Engineering Graduate Fellowship}, AFOSR grant.}

\cvitem{2017}{\textbf{Summa Cum Laude}, Brandeis University.}

\cvitem{2017}{\textbf{Arnold Shapiro Prize in Mathematics}, Brandeis University.}

\cvitem{2017}{\textbf{Michtom Prize in Computer Science}, Brandeis University.}

\cvitem{2016}{\textbf{Phi Beta Kappa}, Brandeis University Chapter, Junior Year Inductee.}

\cvitem{2016}{\textbf{Outstanding Presentation Award}, MAA MathFest 2016.}

\cvitem{2013}{\textbf{Presidential Merit Scholarship}, Brandeis University.}

\cvitem{2013}{\textbf{National Merit Scholarship}, Brandeis University.}



%Papers
\section{Papers}

\cvitem{2025}{\textbf{Categorical Tiling Theory: Constructing Directed Planar Tilings via Edge Reversal.} \textit{Submitted for publication}. With Catherine DiLeo and Preston Sessoms. \hlink{[arXiv:2509.06363]}{https://arxiv.org/abs/2509.06363}}

\cvitem{2025}{\textbf{Additivity and Fiber Sequences for Combinatorial K-Theory.} \textit{Submitted for publication}. With Maru Sarazola. \hlink{[arXiv:2107.07701]}{https://arxiv.org/abs/2107.07701}}

\cvitem{2024}{\textbf{A Combinatorial Construction of Homology via ACGW Categories.} \textit{To appear in Contemporary Mathematics}. With Maru Sarazola and Inna Zakharevich. \hlink{[arXiv:2410.00276]}{https://arxiv.org/abs/2410.00276}}

\cvitem{2024}{\textbf{All Concepts are $\mathbb{C}\mathbf{at}^\#$.} \textit{Submitted for publication}. With Owen Lynch and David Spivak. \hlink{[arXiv:2305.02571]}{https://arxiv.org/abs/2305.02571}}

\cvitem{2024}{\textbf{A Polynomial Construction of Nerves for Higher Categories.} \textit{Submitted for publication}. With David Spivak. \hlink{[arXiv:2405.13157]}{https://arxiv.org/abs/2405.13157}}

\cvitem{2023}{\textbf{Structures in Categories of Polynomials.} \textit{Submitted for publication}. With David Spivak. \hlink{[arXiv:2305.00167]}{https://arxiv.org/abs/2305.00167}}

\cvitem{2023}{\textbf{A Compositional Account of Motifs, Mechanisms, and Dynamics in Biochemical Regulatory Networks.} \textit{Compositionality} 6, \textbf{2}, 2024. With Rebekah Aduddell, James Fairbanks, Amit Kumar, Pablo Ocal, and Evan Patterson. \hlink{[arXiv:2301.01445]}{https://arxiv.org/abs/2301.01445}}

\cvitem{2022}{\textbf{Duoidal Structures for Compositional Dependence.} \textit{Submitted for publication}. With David Spivak. \hlink{[arXiv:2210.01962]}{https://arxiv.org/abs/2210.01962}}

\cvitem{2022}{\textbf{Dynamic Operads, Dynamic Categories: From Deep Learning to Prediction Markets.} \textit{Electronic Proceedings in Theoretical Computer Science}, 2022. With David Spivak. \hlink{[arXiv:2205.03906]}{https://arxiv.org/abs/2205.03906}}

\cvitem{2022}{\textbf{Enrichment of Algebraic Higher Categories.} \textit{Preprint}. \hlink{[arXiv:2205.12235]}{https://arxiv.org/abs/2205.12235}}

\cvitem{2021}{\textbf{Familial Monads as Higher Category Theories.} \textit{Submitted for publication}. \hlink{[arXiv:2111.14796]}{https://arxiv.org/abs/2111.14796}}

\cvitem{2021}{\textbf{Weak Cartesian Properties of Simplicial Sets.} \textit{Journal of Homotopy and Related Structures}, 18, \textbf{477-520}, 2023. With Carmen Constantin, Tobias Fritz, and Paolo Perrone. \hlink{[arXiv:2105.04775]}{https://arxiv.org/abs/2105.04775}}
%Constantin, C., Fritz, T., Perrone, P. et al. Weak cartesian properties of simplicial sets. J. Homotopy Relat. Struct. 18, 477–520 (2023)

\cvitem{2020}{\textbf{Partial Evaluations and the Compositional Structure of the Bar Construction.} \textit{Theory and Applications of Categories}, Vol. 39, No. 11, \textbf{322-364}, 2023. With Carmen Constantin, Tobias Fritz, and Paolo Perrone. \hlink{[arXiv:2009.07302]}{https://arxiv.org/abs/2009.07302}}

\cvitem{2016}{\textbf{Densities of Hyperbolic Cusp Invariants.}  \textit{Proceedings of the American Mathematical Society}, Volume 146, Number 9, \textbf{4073-4089}, 2018. With Colin Adams, Rose Kaplan-Kelly, Michael Moore, Shruthi Sridhar, and Josh Wakefield. \hlink{[arXiv:1701.03479]}{https://arxiv.org/abs/1701.03479}}

\cvitem{2016}{\textbf{specgen: A Tool for Modeling Statecharts in CSP.} \textit{Nasa Formal Methods} \textbf{282}, 2017.  With Chris Casinghino.}

\cvitem{2015}{\textbf{Nonstandard Neutrino Interactions In Supernovae.}\,\textit{Physical Review D} \textbf{94}, 093007, 2016. With C.J. Stapleford, D.J. V\"{a}\"{a}n\"{a}nen, J.P. Kneller, and G.C. McLaughlin. \hlink{[arXiv:1605.04903]}{https://arxiv.org/abs/1605.04903}}



%Teaching
\section{Teaching} 

\cventry{2025}{Differential Equations}{University of Virginia}{Math 3250}{both semesters}{}

\cventry{2024}{Algebraic Topology II}{University of Virginia}{Math 7810}{}{}%{Taught and designed materials for a section of first semester calculus, incorporated active learning techniques, wrote homework problems.}

\cventry{2024}{Linear Algebra}{University of Virginia}{Math 3351}{}{}%{Taught and designed materials for a section of first semester calculus, incorporated active learning techniques, wrote homework problems.}

\cventry{2023}{Differential Geometry}{University of Virginia}{Math 4720}{}{}%{Taught and designed materials for a section of first semester calculus, incorporated active learning techniques, wrote homework problems.}

\cventry{2021}{Calculus I}{Cornell University}{Math 1110}{}{}%{Taught and designed materials for a section of first semester calculus, incorporated active learning techniques, wrote homework problems.}

\cventry{2019}{Applied Linear Algebra (Teaching Assistant)}{Cornell University}{Math 2310}{}{}%{Ran and designed materials for discussion and review sessions.}

\cventry{2018}{Geometric Group Theory (Teaching Assistant)}{Cornell University}{Math 4560}{}{}%{Graded proof based assignments.}

\cventry{2016}{Discrete Math (Teaching Assistant)}{Brandeis University}{COSI 29a}{}{}%{Graded proof based assignments.}

\cventry{2015}{Java Programming (Teaching Assistant, Tutor)}{Brandeis University}{COSI 12b}{}{}%{Graded programming assignments, gave personalized code reviews.}

%\cventry{2015}{Java Programming (Tutor)}{Brandeis University}{COSI 12b}{}{}



\section{Mentorship}

\cventry{2025}{Summer Research Mentor}{University of Virginia}{Triangle algebras}{}{Worked with an undergraduate on developing the theory of a 2-dimensional algebraic structure for composing grids of triangles.}

\cventry{2024-2025}{Peer Mentoring Initiative}{University of Virginia}{Organizer and mentor}{}{Students at different career stages are paired with more experienced students or postdocs for meetups and mentorship.}

\cventry{2024}{REU Mentor}{University of Virginia}{Categorical tiling theory}{}{Lead a group of undergraduates pursuing original research over the summer on a categorical approach to combinatorial geometry.}

\cventry{2023-2025}{Geometry Lab Mentor}{University of Virginia}{Zome geometry group}{}{Lead groups of beginning undergraduates in semester-long exploratory projects.}

\cventry{2023}{Workshop on $(\infty,2)$-Categories}{Online}{}{}{Mentored a student in preparation for an expository talk based on a research paper.}

\cventry{2020-2022}{Directed Reading Program}{Cornell University}{}{}{Mentored three undergraduates in projects on constructive type theory, categorical algebra, and combinatorial topology.}

\cventry{2020}{Julia Robinson Math Festival Volunteer}{Cornell University}{}{}{}%{Helped elementary school students build and analyze hexaflexagons.}




\section{Conference Organizing}

\cvitem{2023}{\textbf{Special Session on Category Theory and Machine Learning.} CALCO, Bloomington.}

%Talks 
\section{Conference Talks}

\cvitem{2025}{\textbf{Directed Tilings of the Euclidean and Hyperbolic Plane.} MAA Spring 2025 DC-MD-VA Section Meeting, Fairfax.}

\cvitem{2024}{\textbf{Homotopy Theory of Double Categories for Algebraic $K$-Theory.} AMS Spring Southeastern Sectional Meeting, Tallahassee.}

\cvitem{2024}{\textbf{CGW Categories.} Workshop on Higher Segal Spaces and their Applications to Algebraic K-Theory, Hall Algebras, and Combinatorics, BIRS.}

\cvitem{2023}{\textbf{Finite Posets as Algebraic Expressions in Duoidal Categories.} Category Theory OctoberFest, Online.}

\cvitem{2023}{\textbf{A Dynamic Monoidal Category for Deep Learning.} CALCO, Bloomington.}

\cvitem{2022}{\textbf{Polynomial Functors for Categorical Open Dynamics.} Joint Math Meetings, Special Session on Applied Category Theory, Boston.}

\cvitem{2022}{\textbf{Double Presheaf Categories via Polynomial Functors.} Virtual Double Categories Workshop, Online.}

\cvitem{2022}{\textbf{Dynamic Operads for Evolving Organizations.} Applied Category Theory, Glasgow.}

\cvitem{2022}{\textbf{Familial Monads for Higher and Lower Category Theory.} Workshop on Polynomial Functors, Online.}

\cvitem{2021}{\textbf{Compositional Structure of Partial Evaluations.} Categories and Companions Symposium, Online.}

\cvitem{2019}{\textbf{Shape Independent Category Theory.} Category Theory OctoberFest, Baltimore.}

\cvitem{2019}{\textbf{Types as Weak $\omega$-Groupoids.} School and Workshop on Univalent Foundations, Birmingham.}

\cvitem{2018}{\textbf{Cell Shapes for Higher Structures.} Young Topologists Meeting, Copenhagen.} 

\cvitem{2016}{\textbf{The Geometry of Knots.} With Shruthi Sridhar. MAA MathFest, Columbus.} 

\cvitem{2016}{ \textbf{Cusp Density: Dense or Knot?} Unknot III, Columbus.}

%\cvitem{2014}{ \textbf{Neutrinos and the Unknown} Museum of Natural Sciences, Raleigh, NC.}


\section{Seminar and Colloquium Talks}

\cvitem{2025}{\textbf{Chain Complexes of Finite Sets and Venn Diagrams for Homological Algebra.} Virginia Commonwealth University Analysis, Logic, and Physics Seminar.}

\cvitem{2023}{\textbf{Combinatorial Homological Algebra and $K$-Theory.} University of Minnesota Topology Seminar.}

\cvitem{2023}{\textbf{Higher Category Theory in $\mathbb{C}\mathbf{at}^\#$.} Topos Institute Colloquium, Online.}

%\cvitem{2021}{\textbf{Algebraic $K$-Theory of Finite Sets via Chain Complexes.} Cornell Topology Seminar.}

\cvitem{2020}{\textbf{Compositional Structure of Partial Evaluations.} MIT Categories Seminar, Online.}

\cvitem{2020}{\textbf{Cubical $\omega$-Categories and Cubical $\Theta$.} MSRI Cubical Sets Seminar, Online.}

\cvitem{2020}{\textbf{Test Category Structure of Cubes.} MSRI Cubical Sets Seminar, Online.}

\cvitem{2020}{\textbf{Constructing Cubes from Semicubes.} MSRI Cubical Sets Seminar, Online.}


%Research Experience
\section{Workshop Participation}

\cventry{2024}{Polynomial Functors at Work}{Topos Institute}{}{}{}

\cventry{2023}{Talbot Workshop}{Online}{Scissors Congruence $K$-Theory, Presenter}{}{}

\cventry{2023}{Math and Metaphysics Symposium}{Austin, TX}{Presenter}{}{}

\cventry{2023}{Finding the Right Abstractions for Healthy Systems}{Bodega Bay, CA}{}{}{}

\cventry{2021}{Equivariant Algebra Seminar}{eCHT}{Presenter}{}{}

%\cventry{2020}{Introductory Workshop: Higher Categories and Categorification}{MSRI}{}{}{}%{TQFT, Trace Methods in K-Theory, Model Independent $(\infty,1)$-Category Theory}

\cventry{2019}{Applied Category Theory Adjoint School \& Workshop}{University of Oxford}{}{}{}%{Project on partial evaluations \& categorical probability, led by Tobias Fritz, Paolo Perrone.}

\cventry{2019}{School \& Workshop on Univalent Foundations}{University of Birmingham}{}{}{}%{Group on formalizing category theory in UniMath.}

\cventry{2018}{Homotopy Theory Summer}{Berlin Mathematical School}{}{}{}%{Equivariant Homotopy Theory \& K-Theory, $\infty$-Categorical $A^1$ Homotopy Theory.}

\cventry{2018}{Talbot Workshop}{Govt. Camp, OR}{Model Independent $\infty$-Category Theory}{}{}%{Model Independent $\infty$-Category Theory.  Mentored by Emily Riehl, Dominic Verity.}

%\cventry{Spring 2017}{Independent Research}{}{}{}{Worked on ongoing research projects with Colin Adams and Daniel Ruberman}

\cventry{2016}{SMALL REU}{Williams College}{Hyperbolic Knot Theory Group}{}{}%{Hyperbolic Knot Theory Group.  Advised by Colin Adams.}

%\cventry{2015}{Internship Project}{Draper Laboratories}{Formal Methods Group}{}{}%{Developed and implemented in Haskell a translation model from statecharts into CSPm}

%\cventry{2015}{Independent Study in Haskell and Type Theory}{Brandeis University}{}{}{Research oriented group study. Supervised by Harry Mairson and Kenneth Foner.}

%\cventry{2014-2015}{Astrophysics Research}{Brandeis University}{}{}{}%{Analyzed plasma jets from AGN via image processing. Advised by David Roberts.}

\cventry{2014}{Computational Astrophysics REU}{North Carolina State University}{}{}{}
%{Analyzed dependence of supernova neutrino oscillations on potential non-standard particle interactions using computer simulation. Advised by James Kneller.}


%%Classes
%\section{Graduate Coursework}
%\cvitem{Cornell}{Algebraic K-theory, Simplicial Structures, Algorithms, Model Theory, Homological Algebra, Programming Language Theory, Kleene Algebras, Topological K-Theory, Type Theory, Hyperbolic Geometry, Algebraic Geometry.}
%\cvitem{Brandeis}{Homotopy Theory, Algebraic Topology (2), Algebra (2), Smooth Manifolds, Differential Geometry, Modal Logic for Computational Linguistics, Statistical Mechanics, General Relativity, Classical Mechanics.}
%\cvitem{IISc}{Computational Geometry \& Topology, Mathematical Quantum Mechanics, Complex Analysis, Measure Theory.}


%Independent Study
\section{Local Seminars}

\cventry{2023-}{Homotopy Theory Group Meeting}{University of Virginia}{Organizer, Presenter}{}{}

\cventry{2023-}{Topology Seminar}{University of Virginia}{Organizer, Presenter}{}{}

\cventry{2024}{Spectra Reading Group}{University of Virginia}{Presenter}{}{}

\cventry{2022-2023}{Berkeley Seminar}{Topos Institute}{Organizer, Presenter}{}{}

\cventry{2018-2022}{Homotopy Group}{Cornell University}{Organizer, Presenter}{}{}

\cventry{2017-2022}{Topology Seminar}{Cornell University}{Presenter}{}{}

\cventry{2020}{Logic Seminar}{Cornell University}{Presenter}{}{}

\cventry{2019}{``What is...?'' Seminar}{Cornell University}{Organizer}{}{}

\cventry{2018}{$\infty$-Category Theory Reading Group}{Cornell University}{Presenter}{}{}

\cventry{2018}{Homotopy Type Theory Group}{Cornell University}{Organizer, Presenter}{}{}

\cventry{2017-2022}{Olivetti Club}{Cornell University}{Presenter}{}{}

\cventry{2016-2017}{Floer Homology Group}{Brandeis University}{}{}{}

\cventry{2015}{Haskell and Type Theory Group}{Brandeis University}{}{}{}



%%Extracurricular
%\section{Extracurricular}
%
%\cventry{2018-2022}{Incoming Graduate Student Mentor}{Cornell University}{}{}{}
%
%\cventry{2018-2022}{Class Representative}{Cornell University}{}{}{}%{Represent the concerns and interests of my PhD class to the math department.}
%
%\cventry{2018}{Math Department Spring Concert Organizer}{Cornell University}{}{}{}
%
%\cventry{2018}{Guest Speaker on College Math}{Walt Whitman High School}{}{}{}
%
%\cventry{2016-2017}{Math Club Founder and President}{Brandeis University}{}{}{}%{Organized meetings and led newly formed math club to student union recognition.}
%
%\cventry{2015-2017}{Undergraduate Mathematics Department Representative}{Brandeis University}{}{}{}%{
%%Acted as student advisor and undergaduate liaison in math department. \\[0.2em]
%%Organized events and programs for undergraduates in mathematics.}% \\
%%\hspace*{1em} {\tiny \textbullet} \textbf{Directed Reading Program.} Grad student supervised independent study projects.\\
%%\hspace*{1em} {\tiny \textbullet} \textbf{Pi Day Extravaganza.} Interdisciplinary library display and presentations on pi.\\
%%\hspace*{1em} {\tiny \textbullet} \textbf{Welcome and Advising Sessions.} For undergrads with faculty and grad students.}
%


%Computer Skills
%\section{Computer Skills}
%\cvitem{Software}{Sage, Maple, Mathematica, MatLab, \LaTeX}
%\cvitem{Languages}{Python, Java, Unix/Linux}

\end{document}